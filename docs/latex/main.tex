\documentclass[10pt,a4paper]{jarticle}

\usepackage[dvipdfm]{graphicx}
\usepackage[]{amsmath}
\usepackage{enumerate}
\usepackage{latexsym}
\usepackage{times}
\usepackage[multi,deluxe]{otf}
\usepackage{subfiles}
\usepackage{caption}
\usepackage{subcaption}

\graphicspath{ {images/} }

\topmargin        0mm
\oddsidemargin    -0.4mm
\evensidemargin   -0.4mm
\textwidth        160mm
\textheight       237mm
\columnsep        7mm
\headheight       0mm
\footskip         10mm
\headsep          0mm
\partopsep        0mm
\textfloatsep     3mm
\intextsep        3mm

\begin{document}
\begin{twocolumn}
    [
     \begin{flushright}
      九州大学 CPC Lab 研究会\\
      2021年度 第13回 研究会資料 TR2021-013
     \end{flushright}
     
     \begin{center}
      \begin{Large}
       \begin{bf}
           プロセッサ設計演習
       \end{bf}
      \end{Large}
      
      \begin{large}
       ムハマド ナウファル \\
       九州大学 工学部 電気情報工学科 4年\\
       2020年7月05日
      \end{large}
     \end{center}
    ]

    \section{はじめに}
        \subfile{sections/intro.tex}

    \section{プロセッサの仕様}
        \subfile{sections/specs.tex}

    \section{プロセッサの構成}
        \subfile{sections/processor.tex}

    \section{検証結果}
        \subfile{sections/eval.tex}

    \section{改善点}
        \subfile{sections/improv.tex}

    \section{改善できる点}
        \subfile{sections/upgradable.tex}

    \section{まとめ}
        \subfile{sections/summary.tex}

\end{twocolumn}
\end{document}

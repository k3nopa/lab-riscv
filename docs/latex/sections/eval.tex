\documentclass[../main.tex]{subfiles}

\begin{document}
    実装したプロセッサをcadence社のツールNC-Verilogを用いて性能検証する前に、
    正常な動作ができるように、シミュレーションを行い、
    用意されたテストプログラムを利用し、
    全ての命令が正しく動作することを確認した。
    シミュレーションをする際、1クロックサイクルはあたり10nsとして実行した.

    以上の機能検証が終わり、
    性能評価のために用いたテストプログムラ、MiBenchである。
    先に、用意されたベンチマークプログラムは
    正しい出力が得られるかどかうかのtestプログラムで判断する。
    表\ref{table:benchmark}に示すように、
    bitcount、dijkstra、stringsearchの3種類から、それぞれtest、small、largeの3種類を持ち、
    ただし、largeには問題があるため、
    ここで、largeの検証結果を出さないことにする。
    総クロック数を求め、性能評価の基準として扱う。

    論理合成にはDesign Compilerというツールを用いて、
    最小周期の制約、面積、および消費電力を求めた。

    \begin{table}[h]
        \centering
        \caption{ベンチマークの実行結果}
        \label{table:benchmark}
        \begin{tabular}{ |c|c| }
         \hline
         ベンチマーク名 & 実行時間[ns]\\
         \hline
         bitcnts:test & 2000285 \\
        \hline
         bitcnts:small & 1417166995 \\
        \hline
         dijkstra:test & 107521545 \\
        \hline
         dijkstra:small & 985645445 \\
        \hline
         stringsearch:test & 351885 \\
        \hline
         stringsearch:small & 3007895 \\
        \hline
        \end{tabular}
    \end{table}

    \begin{table}[h]
        \centering
        \caption{論理合成の結果}
        \label{table:synthesis}
        \begin{tabular}{ |c|c| }
        \hline
        最大遅延時間[ns] & 5.15 \\
        \hline
        面積[$\mu$$m^2$] & 265559 \\
        \hline
        消費電力[mW] & 4.7185 \\
        \hline
        \end{tabular}
    \end{table}
\end{document}
